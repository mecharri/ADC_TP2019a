Guía para la realización del Trabajo Práctico

\begin{itemize}
\item Definir el tipo de filtro, fo, Q, los polos y los ceros (fo es la frecuencia de corte para los filtros pasa bajo y pasa altos y la frecuencia central para los filtros pasa banda y rechazo de banda)
\item Realizar el diagrama de Bode (módulo y fase) con Octave/Matlab
\item Obtener la respuesta al escalón y la respuesta al impulso con Octave/Matlab
\item Obtener la respuesta en Octave/Matlab cuando la excitación es una onda cuadrada de las siguientes frecuencias: fo/10; fo; 10*fo 
\item Encontrar un circuito con amplificadores operacionales que cumpla con la transferencia propuesta
\item Definir los valores de los componentes, utilizar valores normalizados  tanto para los capacitores (preferentemente en la serie del 10%) como los resistores (preferentemente en la serie del 10% o eventualmente del 5% o del 1%). Definir primero el valor de los capacitores y luego el de los resistores (en casos excepcionales colocar resistores ajustables o preset). Obtener la transferencia con los valores normalizados de los componentes elegidos.
\item Realizar nuevamente los diagramas de Bode y obtener la respuesta al escalón con los valores normalizados elegidos para el circuito. Comparar con lo obtenido en los puntos 1 y 2.
\item Realizar la simulación del circuito con SPICE, comparar el diagrama de Bode, la respuesta al escalón y la respuesta a la onda cuadrada (con las mismas especificaciones del punto 4) con la obtenida con Octave/Matlab.
\item Realizar un esquema del circuito con todos sus componentes e indicando los terminales del circuito integrado que se conectan. Armar el filtro en un protoboard (aunque se recomienda armar el filtro en un circuito impreso multiperforado o diseñado específicamente).
\item Realizar la medición del filtro utilizando fuentes de alimentación externa, generador de funciones y osciloscopio. 
	\subitem Medir la respuesta del filtro con excitación senoidal en el rango de frecuencias fo/10 hasta 10*fo en tercios de octava y en el caso de filtros pasabanda o atenuador de banda medir por lo menos 10 puntos entre fo/2 y 2fo separados equitativamente en escala logarítmica 
	\subitem Definir el procedimiento para medir las características principales del filtro: frecuencia/as de corte a -3dB, frecuencia de intersección de las asíntotas, pendientes de las asíntotas de la respuesta en frecuencia. 
	\subitem Realizar las mediciones. Tabular las mediciones y realizar un gráfico con los valores obtenidos.
	\subitem Obtener la respuesta del filtro a una excitación de onda cuadrada (con las características especificadas en el punto 4)
	\subitem Comparar los resultados de la medición con las simulaciones. Conclusiones.
	\subitem Realizar un diagrama de interconexión con el instrumental utilizado y sus características. Agregar fotos del banco de mediciones y de las impresiones de pantalla del osciloscopio obtenidas en el punto d)
\item Realizar un informe con toda la información solicitada en los puntos anteriores. Conclusiones.
\end{itemize}

\begin{equation}
H(s) = \frac{4.564\cdot 10^{8}}{s^2 + s\cdot 3.021 \cdot 10^{4} + 4.564 \cdot 10^{8}}
\end{equation}

\begin{equation}
H(s) = \frac{ 2.528 \cdot 10^{9}}{s^2 + s\cdot 7.108 \cdot 10^{4} + 2.528 \cdot 10^{9}}
\end{equation}

\begin{equation}
H(s) = \frac{ 2.26 \cdot 10^{8}}{s^2 + s \cdot 1.791 \cdot 10^{4} + 2.394 \cdot 10^{8}}
\end{equation}

\begin{equation}
H(s) = \frac{ \cdot 10^{}}{s^2 + s \cdot  \cdot 10^{} +  \cdot 10^{}}
\end{equation}

\begin{equation}
H(s) = \frac{ \cdot 10^{}}{s^2 + s \cdot  \cdot 10^{} +  \cdot 10^{}}
\end{equation}

\begin{equation}
H(s) = \frac{ \cdot 10^{}}{s^2 + s \cdot  \cdot 10^{} +  \cdot 10^{}}
\end{equation}

\begin{equation}
H(s) = \frac{ \cdot 10^{}}{s^2 + s \cdot  \cdot 10^{} +  \cdot 10^{}}
\end{equation}

\begin{equation}
H(s) = \frac{ \cdot 10^{}}{s^2 + s \cdot  \cdot 10^{} +  \cdot 10^{}}
\end{equation}

\begin{equation}
H(s) = \frac{ \cdot 10^{}}{s^2 + s \cdot  \cdot 10^{} +  \cdot 10^{}}
\end{equation}

\begin{equation}
H(s) = \frac{ \cdot 10^{}}{s^2 + s \cdot  \cdot 10^{} +  \cdot 10^{}}
\end{equation}

\begin{equation}
H(s) = \frac{ \cdot 10^{}}{s^2 + s \cdot  \cdot 10^{} +  \cdot 10^{}}
\end{equation}

\begin{equation}
H(s) = \frac{ \cdot 10^{}}{s^2 + s \cdot  \cdot 10^{} +  \cdot 10^{}}
\end{equation}

\begin{equation}
H(s) = \frac{ \cdot 10^{}}{s^2 + s \cdot  \cdot 10^{} +  \cdot 10^{}}
\end{equation}

\begin{equation}
H(s) = \frac{ \cdot 10^{}}{s^2 + s \cdot  \cdot 10^{} +  \cdot 10^{}}
\end{equation}

\begin{equation}
H(s) = \frac{ \cdot 10^{}}{s^2 + s \cdot  \cdot 10^{} +  \cdot 10^{}}
\end{equation}

\begin{equation}
H(s) = \frac{ \cdot 10^{}}{s^2 + s \cdot  \cdot 10^{} +  \cdot 10^{}}
\end{equation}

\begin{equation}
H(s) = \frac{ \cdot 10^{}}{s^2 + s \cdot  \cdot 10^{} +  \cdot 10^{}}
\end{equation}

\begin{equation}
H(s) = \frac{ \cdot 10^{}}{s^2 + s \cdot  \cdot 10^{} +  \cdot 10^{}}
\end{equation}

\begin{equation}
H(s) = \frac{ \cdot 10^{}}{s^2 + s \cdot  \cdot 10^{} +  \cdot 10^{}}
\end{equation}

\begin{equation}
H(s) = \frac{ \cdot 10^{}}{s^2 + s \cdot  \cdot 10^{} +  \cdot 10^{}}
\end{equation}
