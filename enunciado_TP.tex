\documentclass[a4papper, 12pt, spanish]{article}
\textheight=23cm
\textwidth=17cm
\topmargin=-1cm
\oddsidemargin=0cm
\parindent=0mm
\usepackage{amsmath,amssymb,amsfonts,latexsym,cancel} %Soporte para s�mbolos y font matem�ticos
\usepackage{graphicx}			      %Soporte para gr�ficos.
\DeclareGraphicsExtensions{.png}
\usepackage[latin1]{inputenc} % Caracteres con acentos.
\usepackage[spanish]{babel} % Caracteres con acentos.
\usepackage{amsmath,amssymb,amsfonts,latexsym,cancel} %Paquetes de fuentes adicionales
\usepackage{hyperref}
\spanishdecimal{,}

% Colores:
\usepackage{color}
\definecolor{gray97}{gray}{.97}
\definecolor{gray75}{gray}{.75}
\definecolor{gray45}{gray}{.45}

%%%%%%%%%%%%%%%%%%%%%%%%%%%%%%%%%%%%
%% LISTINGS (Inclusi�n de c�digo) %%
%%%%%%%%%%%%%%%%%%%%%%%%%%%%%%%%%%%%
\usepackage{listings}
\lstset{
    language            =   Matlab,
    frame               =   Ltb,
    framerule           =   0pt,
    aboveskip           =   0.5cm,
    framextopmargin     =   3pt,
    framexbottommargin  =   3pt,
    framexleftmargin    =   0.4cm,
    framesep            =   0pt,
    rulesep             =   0.4pt,
    backgroundcolor     =   \color{gray97},
    rulesepcolor        =   \color{black},
    stringstyle         =   \ttfamily,
    showstringspaces    =   false,
    basicstyle          =   \small\ttfamily,
    commentstyle        =   \color{gray45},
    keywordstyle        =   \bfseries,
    numbers             =   left,
    numbersep           =   15pt,
    numberstyle         =   \tiny,
    numberfirstline     =   false,
    breaklines          =   true,
    breakatwhitespace   =   true,
}



\begin{document}

\begin{titlepage}
\vskip2.5cm
\begin{center}
\begin{tabular}{p{1cm} p{11cm}  p{2.5cm}}

\includegraphics[scale=.35]{./figs/Logo_FIUBA.jpg}

&
\large{
\begin{center}
\sc An�lisis de circuitos\\
\sc 86.04/6606\\
\sc Turno 01
\end{center}
} &
\end{tabular}
\end{center}
\vskip3cm
\begin{center}
\Large {\textbf{
Enunciado de trabajo pr�ctico}}\\
\end{center}
\vskip3cm
\begin{center}

\end{center}
\vskip3.5cm
\begin{flushright}
\begin{tabular}{r l}
\large{\textbf{Primer cuatrimestre 2019}}\\

\end{tabular}
\end{flushright}
\vskip3.5cm

\end{titlepage}

\section{Gu�a para la realizaci�n del Trabajo Pr�ctico}

El trabajo pr�ctico se entrega y defiende en forma individual. A cada estudiante se le asignar� una transferencia al azar a partir de la lista de la secci�n \ref{SECTION::TRANSFERENCIAS}. Dentro del cuatrimestre se asignar� un d�a para la entrega del informe y un d�a para la medici�n.


A partir de la transferencia asiganada se pide:

\begin{enumerate}
\item Definir: el tipo de filtro, $f_0$, $\omega_0$\footnote{$f_0$ es la frecuencia de corte para los filtros pasa bajo y pasa altos y la frecuencia central para los filtros pasa banda y rechazo de banda, $\omega_0$ es la correspondiente pulsaci�n}, $Q$, los polos y los ceros \label{ITEM::ITEM1}
\item Realizar el diagrama de Bode (m�dulo y fase) con Octave/Matlab \label{ITEM::ITEM2}
\item Obtener la respuesta al escal�n y la respuesta al impulso con Octave/Matlab
\item Obtener la respuesta en Octave/Matlab cuando la excitaci�n es una onda cuadrada de las siguientes frecuencias: $f = \frac{f_0}{10}$; $f = f_0$; $f = 10\cdot f_0 $ \label{ITEM::ITEM4}
\item Encontrar un circuito con amplificadores operacionales que cumpla con la transferencia propuesta
\item Definir los valores de los componentes, utilizar valores normalizados tanto para los capacitores (serie del $10\%$, serie $E12$) como los resistores (serie del $1\%$, serie $E96$). Definir primero el valor de los capacitores y luego el de los resistores. Obtener la transferencia con los valores normalizados de los componentes elegidos. Los valores de resistencias son los comprendidos entre los rangos $4.7k\Omega$ y $1.6M\Omega$, para los capacitores los ocmprendidos entre $1nF$ y $470nF$, de no poder utilizar estos valores se debe demostrar que las corrientes a la salida del operacional no superen los $5mA$.
\item Realizar nuevamente los diagramas de Bode y obtener la respuesta al escal�n con los valores normalizados elegidos para el circuito. Comparar con lo obtenido en los puntos \ref{ITEM::ITEM1} y \ref{ITEM::ITEM2}.
\item Realizar la simulaci�n del circuito con SPICE, comparar el diagrama de Bode, la respuesta al escal�n y la respuesta a la onda cuadrada (con las mismas especificaciones del punto 4) con la obtenida con Octave/Matlab.
\item Realizar un esquema del circuito con todos sus componentes e indicando los terminales del circuito integrado que se conectan. Armar el filtro en un protoboard (aunque se recomienda armar el filtro en un circuito impreso multiperforado o dise�ado espec�ficamente).
\item Realizar la medici�n del filtro utilizando fuentes de alimentaci�n externa, generador de funciones y osciloscopio. 
\begin{enumerate}
	\item Medir la respuesta del filtro con excitaci�n senoidal en el rango de frecuencias $\frac{f_0}{10}$ hasta $ 10\cdot f_0 $ en tercios de octava y en el caso de filtros pasabanda o atenuador de banda medir por lo menos $10$ puntos entre $\frac{f_0}{2}$ y $2\cdot f_0$ separados equitativamente en escala logar�tmica 
	\item Definir el procedimiento para medir las caracter�sticas principales del filtro: frecuencia/as de corte a $-3dB$, frecuencia de intersecci�n de las as�ntotas, pendientes de las as�ntotas de la respuesta en frecuencia. 
	\item Realizar las mediciones. Tabular las mediciones y realizar un gr�fico con los valores obtenidos.
	\item Obtener la respuesta del filtro a una excitaci�n de onda cuadrada (con las caracter�sticas especificadas en el punto \ref{ITEM::ITEM4})\label{ITEM::ITEMd}
	\item Comparar los resultados de la medici�n con las simulaciones. Conclusiones.
	\item Realizar un diagrama de interconexi�n con el instrumental utilizado y sus caracter�sticas. Agregar fotos del banco de mediciones y de las impresiones de pantalla del osciloscopio obtenidas en el punto \ref{ITEM::ITEMd})
\end{enumerate}
\item Realizar un informe con toda la informaci�n solicitada en los puntos anteriores.
\item Conclusiones.
\end{enumerate}

\section{Transferencias \label{SECTION::TRANSFERENCIAS}}

$$H_1(s) = \frac{4.564\cdot 10^{8}}{s^2 + s\cdot 3.021 \cdot 10^{4} + 4.564 \cdot 10^{8}}$$

$$H_2(s) = \frac{ 2.528 \cdot 10^{9}}{s^2 + s\cdot 7.108 \cdot 10^{4} + 2.528 \cdot 10^{9}}$$

$$H_3(s) = \frac{ 2.26 \cdot 10^{8}}{s^2 + s \cdot 1.791 \cdot 10^{4} + 2.394 \cdot 10^{8}}$$

$$H_4(s) = \frac{ 6.207\cdot 10^{8}}{s^2 + s \cdot 2.759 \cdot 10^{4} + 6.207 \cdot 10^{8}}$$

$$H_5(s) = \frac{ s^2}{s^2 + s \cdot 2666 + 3.553 \cdot 10^{6}}$$

$$H_6(s) = \frac{ s^2}{s^2 + s \cdot 444 + 9.869 \cdot 10^{4}}$$

$$H_7(s) = \frac{ 0.9441 \cdot s^2}{s^2 + s \cdot 590.8 + 2.604 \cdot 10^{5}}$$

$$H_8(s) = \frac{ s^2}{s^2 + s \cdot 3510 + 1.004 \cdot 10^{7}}$$

$$H_9(s) = \frac{ s^2 }{s^2 + s \cdot 13333 + 8.889 \cdot 10^{7}}$$

$$H_{10}(s) = \frac{ 1847 \cdot s}{s^2 + s \cdot 1847 + 1.567 \cdot 10^{7}}$$

$$H_{11}(s) = \frac{ 2928 }{s^2 + s \cdot 2928 + 3.948 \cdot 10^{7}}$$

$$H_{12}(s) = \frac{ 4649\cdot s}{s^2 + s \cdot 4649 + 9.95 \cdot 10^{7}}$$

$$H_{13}(s) = \frac{ s \cdot 207}{s^2 + s \cdot 207 + 4.562 \cdot 10^{6}}$$

$$H_{14}(s) = \frac{ 628 }{s^2 + s \cdot 628 + 3.948 \cdot 10^{7}}$$

$$H_{15}(s) = \frac{ 2450 \cdot s}{s^2 + s \cdot 2450 + 6317 \cdot 10^{8}}$$

$$H_{16}(s) = \frac{ s^2 +3.948 \cdot 10^{5}}{s^2 + s \cdot 62.8 + 3.948 \cdot 10^{5}}$$

$$H_{17}(s) = \frac{ s^2 + 3.948\cdot 10^{5}}{s^2 + s \cdot 628 + 3.948 \cdot 10^{7}}$$

%%$$H_{18}(s) = \frac{ \cdot 10^{}}{s^2 + s \cdot  \cdot 10^{} +  \cdot 10^{}}$$
%%
%%$$H(s) = \frac{ \cdot 10^{}}{s^2 + s \cdot  \cdot 10^{} +  \cdot 10^{}}$$
%%
%%$$H(s) = \frac{ \cdot 10^{}}{s^2 + s \cdot  \cdot 10^{} +  \cdot 10^{}}$$



\end{document}
